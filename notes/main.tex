\documentclass[]{article}
\usepackage{fullpage}

\usepackage[utf8]{inputenc}
\usepackage[T1]{fontenc}
\usepackage{microtype}

\usepackage{amssymb}
\usepackage{amsfonts}
\usepackage{amsmath}
\usepackage{amsthm}
\usepackage{mathrsfs}

\usepackage{todonotes}
\usepackage{graphicx}
\usepackage{subcaption}
\usepackage{showlabels}
\graphicspath{{res/}}

\usepackage[english]{babel}
\usepackage{blindtext}

\usepackage[authoryear,round,longnamesfirst]{natbib}
\bibliographystyle{unsrtnat}

\RequirePackage[l2tabu, orthodox]{nag}
\usepackage[all,warning]{onlyamsmath}

\usepackage{hyperref}

\newcommand{\tnorm}{\underline{t}_{\land}}
\newcommand{\snorm}{\underline{s}_{\lor}}

\title{Fuzzy Logic}
\author{Santiago Hincapie-Potes}
\date{\today}

\begin{document}
\maketitle


\section{Fuzzy sets}
The main idea is to model imprecise concepts, such as youngness or tallness.
Although they are everyday concepts, there isn't a formal definition of their
real meaning due to the fact that there is no clear border that delimits them.\\
Although at first glance it may not seem particularly useful, it can be an
interesting tool for modeling imprecise dependencies (e.g.\ rules) for instance
if Temperature is ``low'' and Oil is ``cheap'' then crank up the heating system.

\subsection{Crisp Sets}
Crisp sets A.K.A. Classical sets is one that can be described by the elements that
belong it. It is possible to describe any set using a characteristic function:
\[ m_{A}(x) := \begin{cases}
      1 & x\in A \\
      0 & x\notin A
   \end{cases} \qquad m_{A}(x) \in \{0, 1\}
\]
\paragraph{Example} Consider the following set
\( A = \{x \in \mathbb{R} \mid a \leq x \leq b \}\),
the corresponding characteristic function can be seen in~\ref{fig:crispset}

\begin{figure}[ht!]
  \centering
  \includegraphics[width=8cm]{crisp_set}
  \caption{Plot of the characteristic function of a crisp set example\label{fig:crispset}}
\end{figure}

\subsection{Fuzzy Sets~\citep{Zadeh1965}}
A fuzzy set is a class of objects with a continuum of grades of membership,
as stated above, it tries to model classes of concepts in which the notion of
membership is ambiguous. It is possible to describe any fuzzy set using a
so-called membership function, which is very similar to the characteristic
function used to the crisp sets but now this function assigns values in the
interval $[0, 1]$, so now it is possible to describe classes with intermediate
degrees of membership

\paragraph{Example} Consider the following fuzzy set
$ \hat{A} = x \text{ is roughly in }[a, b] $,
the corresponding characteristic function can be seen in~\ref{fig:fuzzyset}\\
\begin{figure}[ht!]
  \centering
  \includegraphics[width=8cm]{fuzzy_set}
  \caption{Plot of the membership function of a fuzzy set example\label{fig:fuzzyset}}
\end{figure}

So basically, numbers very far from the interval simply do not belong to the set.
Numbers in the interval strongly belong to the set. Numbers close to the extremes
of the interval (i.e.\ roughly in $[a, b]$) belong to the set with a certain degree
of membership between 0 and 1.

\subsection{Linguistic Variables~\citep{Zadeh1975}}
A linguistic variable is a variable whose values (interpretation) are natural
language expressions.~For example, \textit{Age} is a linguistic variable if its
values are \textit{young, old, etc.}, rather than $20, 21, 22, \dots$.
Since words, in general, are less precise than numbers, the concept of a
linguistic variable serves the purpose of providing a means of approximate
characterization of phenomena which are too complex or too ill-defined to be
amenable to description in conventional quantitative terms. \\
More formally a linguistic variable can be characterized by the triplet
$(\mathscr{X}, T(\mathscr{X}), U)$ in which $\mathscr{X}$ is the name of
the variable i.e. \textit{Age}; $T(\mathscr{X})$ is the \textit{term-set} of
$\mathscr{X}$, that is the collection of its linguistic values, i.e.
\textit{young, old, etc.}; $U$ the universe of discourse, the range in
which the variable would take numerical values i.e. As age can take
values between 0 and 200, $U = \{x \in \mathbb{R} \mid 0\leq x\leq 200\}$.\\
The meaning of the value $X$ of a linguistic variable can be characterized by
a fuzzy sets in which, its membership function $\mu: U\to [0, 1]$ tells how
compatible are $X$ to a certain numerical value in $U$. To characterize the
meaning of a linguistic variable, the meaning of a representative subset of its
values is typically used.

\paragraph{Example}
\begin{itemize}
  \item Consider the linguistic variable \textit{Age}, in~\ref{fig:age} we can
    see the graph of the fuzzy sets that represents \textit{young} and
    \textit{old}
    \begin{figure}[ht!]
      \centering
      \includegraphics[width=8cm]{age}
      \caption{Plot of meaning of young and old\label{fig:age}}
    \end{figure}
  \item Consider the linguistic variable \textit{Size}, in~\ref{fig:size} we can
    see the graph of the fuzzy sets that represents \textit{small, medium} and
    \textit{tall}
    \begin{figure}[ht!]
      \centering
      \includegraphics[width=8cm]{size}
      \caption{Plot of meaning of small, medium, tall\label{fig:size}}
    \end{figure}
\end{itemize}

\subsubsection*{Context is critical}
Ambiguity is part of human language and it is this that allows to express
incredibly complex ideas with a relatively small number of bits, linguists
agree that the context in which a word appears is essential to resolve the
possible ambiguities associated with the meaning of a word~\citep{Bransford1972}
On a similar way, the meaning of a linguistic variable should depends on the
context in which it is being used.
\paragraph{Example}
The notion of tallness strongly depends on the context in which is being used, a
small person in the Netherlands may not be that small in Yemen, even more, a
small basketball player may be a huge jockeys rider, in~\ref{fig:context} the
comparison of the fuzzy sets that represents this last scenario is shown

\begin{figure}[ht!]
\centering
\begin{subfigure}{.5\textwidth}
  \centering
  \includegraphics[width=7cm]{basketball}
  \caption{Size of basketball player}
\end{subfigure}%
\begin{subfigure}{.5\textwidth}
  \centering
  \includegraphics[width=7cm]{jockeys}
  \caption{Size of jockeys riders}
\end{subfigure}
\caption{Comparison of the meaning of Size on different sports\label{fig:context}}
\end{figure}

\subsection{Membership functions}
The membership function is a key concept in fuzzy logic as it represents the
degree of truth of a fuzzy set. Beside in theory any mapping between $U$ and
$[0, 1]$ can be used as membership function, the fuzzy logic community has
adopted a relatively low number of parameterizable membership functions with
which it is possible to model a large number of scenarios, in~\ref{fig:membership}

\begin{figure*}
  \centering
  \begin{subfigure}[b]{0.475\textwidth}
    \centering
    \includegraphics[width=\textwidth]{tropezoid}
    \caption{trapezoid<a, b, c, d>\label{fig:trapezoid}}
  \end{subfigure}
  \hfill
  \begin{subfigure}[b]{0.475\textwidth}
    \centering
    \includegraphics[width=\textwidth]{triangular}
    \caption{triangle<a, b, c> = traeozoid<a, b, b, c>\label{fig:triangular}}
  \end{subfigure}
  \vskip\baselineskip
  \begin{subfigure}[b]{0.475\textwidth}
    \centering
    \includegraphics[width=\textwidth]{gaussian}
    \caption{gaussian<mean, std>\label{fig:gaussian}}
  \end{subfigure}
  \hfill
  \begin{subfigure}[b]{0.475\textwidth}
    \centering
    \includegraphics[width=\textwidth]{singleton}
    \caption{singleton: (a, 1) and (b, 0.5)\label{fig:singleton}}
  \end{subfigure}
  \caption{some common membership functions with their respective
    parameterization\label{fig:membership}}
\end{figure*}

There are several properties associated with a membership function, some of the most important
(in fuzzy logic context) are:
\begin{itemize}
  \item{\textbf{Support} elements having non-zero degree of membership}
  \item{\textbf{Core} Set with elements having degree of 1}
  \item{\textbf{$\alpha$-cut} set of elements with degree $\geq\alpha$}
  \item{\textbf{Height} Maximum degree of membership}
\end{itemize}

In~\ref{fig:concepts} these concepts are exemplified.

\begin{figure}[ht!]
  \centering
  \includegraphics[width=8cm]{concepts} % TODO: improve plot
  \caption{Diagram with annotated properties\label{fig:concepts}}
\end{figure}

\subsection{Operators on Fuzzy Sets}
As in classic sets, the fuzzy sets can be operated between them, unfortunately,
unlike the first, there is no a convincing arguments for something like the
``right choice'' of our operations~\citep{Zadeh1965}. Of course, the problem in
the pas has been discussed from different points of view e.g. \citet{Bellman1973},
\citet{Yager1979}, \citet{Giles1979}. Beside those motivational discussions for
the choice of ``right'' operators for fuzzy sets different families of operations
have been discussed by e.g.\citet{Yager1980}, \citet{Dombi1982}, \citet{Weber1983},
all of which proved to be special cases of t-norms~\citep{Schweizer1961}, this is
mainly due the t-norm generalize the usual conjunction operator and hence could
be used to define intersection operation for fuzzy sets, correspondingly his dual,
the t-conorms (AKA s-norm) generalize the usual disjunction operator and thus
could be used to define union operation for fuzzy sets.
\subsubsection{t-norm~\citep{Schweizer1961}}
A t-norm (triangular norm) is a function $\tnorm: [0, 1] \times [0, 1] \to [0, 1] $
which satisfies the following properties:
\begin{enumerate}
  \item \textbf{Commutativity}: $\tnorm(a, b) = \tnorm(b, a)$
  \item \textbf{Monotonicity}: $\tnorm(a, b) \leq \tnorm(c, d)\text{ if } a \leq c\ \&\ b \leq d$
  \item \textbf{Associativity}: $\tnorm(a, \tnorm(b, c)) = \tnorm(\tnorm(a, b), c)$
  \item The number 1 acts as neural element: $\tnorm(a, 1) = a$
\end{enumerate}
The boolean conjunction is both commutative and associative, its generalization
must have these operational properties. The monotonicity property ensures that
the degree of truth of conjuction does not decrease if the thuth values of
conjucts increase. The requirement of 1 as neutral element corresponds to the
interpretation of 1 as true.
\subsubsection{Negation}
The truth function of negation has to be non-incresing (and assing 0 to 1 and
vice versa); the function $1 - x$ is the best known candidate.
\subsubsection{t-conorm}
Using the De morgan duality is possible to define a t-corom $\snorm$ with respect
with respect to a negation operation. This new $\snorm$ would generalize the usual
disjunction operation. More specific

$$ \snorm(u, v) = 1 - \tnorm(1 - u, 1 - v) $$
$$ \tnorm(u, v) = 1 - \snorm(1 - u, 1 - v) $$
\paragraph{Gödel norm}
It occurs in most t-norm based fuzzy logics as the standard semantics for conjunction
$$ \tnorm(x, y) = \min{(x, y)} $$
$$ \snorm(x, y) = \max{(x, y)} $$

This ``and'' must be interpreted in a ``hard'' sense, that is, we do not allow
any tradeoff between $x$ and $y$ so long as $x > y$ or vice-versa. For instance,
if $x = 0.8$ and $x = 0.5$, then $\tnorm(x, y) = 0.5$ so long as $x > 0.5$.
\paragraph{Product norm}
$$ \tnorm(x, y) = x \cdot y $$
$$ \snorm(x, y) = x + y - x \cdot y $$

\paragraph{Lukasiewicz norm}
$$ \tnorm(x, y) = \max{(0, x + y - 1)} $$
$$ \snorm(x, y) = \min{(1, a + b)} $$


\subsubsection{Classic vs Fuzzy logic}
\section{Fuzzy Rule System}

\bibliography{main}

\end{document}
